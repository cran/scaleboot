% scaleboot/vignette.tex shimo@is.titech.ac.jp
% $Id: usesb.Rnw,v 1.3 2006/10/30 02:22:10 shimo Exp $
%%\VignetteIndexEntry{Using scaleboot}
\def\documentid{\getTime-stamp: "2006-10-30 11:16:44 shimo"}
\def\getTime-stamp: "#1"{#1}

\documentclass[a4paper]{amsart}
\usepackage{url}

\def\baselinestretch{1.0} 
\usepackage{/usr/share/R/share/texmf/Sweave}
\begin{document}


\title{Multiscale Bootstrap using scaleboot package}
\author{Hidetoshi Shimodaira}
\thanks{This package vignette is included in {\tt scaleboot}
 available from CRAN. The source file is usesb.Rnw (\documentid).}
\address{Department of Mathematical and Computing Sciences\\
Tokyo Institute of Technology\\
2-12-1 Ookayama, Meguro-ku, Tokyo 152-8552, Japan}
\email{shimo@is.titech.ac.jp}


\maketitle

\section{Introduction}

The {\tt scaleboot} is an add-on package of R. This is for calculating
approximately unbiased (AU) $p$-values for a general problem from a
set of multiscale bootstrap probabilities (BPs). Scaling is equivalent
to changing the sample size of dataset in bootstrap resampling. We
compute BPs at several scales, from which a very accurate $p$-value is
calculated (Shimodaira 2002). This multiscale bootstrap method has
been implemented in {\tt CONSEL} software (Shimodaira and Hasegawa
2001) for phylogenetic inference and {\tt pvclust} package (Suzuki and
Shimodaira 2006) for hierarchical clustering. The thrust of {\tt
scaleboot} package is to calculate an improved version of AU
$p$-values which are justified even for hypotheses with nonsmooth
boundaries (Shimodaira 2006).

The basic usage of this package is illustrated in a simple example
below.  Then real applications in hierarchical clustering and
phylogenetic inference are shown later.

\section{Install}

{\tt scaleboot} is easily installed from CRAN online. Windows users
may use the pull-down menu for install and just choose
``scaleboot''. Otherwise, run R on your computer and type
\begin{Schunk}
\begin{Sinput}
> install.packages("scaleboot")
\end{Sinput}
\end{Schunk}
You can also download the package file from the URL below, and install
manually.

\url{http://www.is.titech.ac.jp/~shimo/prog/scaleboot/}


\section{Simple Example}

\subsection{Simulation Data}

We first generate a simulation dataset.
\begin{Schunk}
\begin{Sinput}
> simdata <- function(n, y, sd) {
+     m <- length(y)
+     x <- matrix(rnorm(m * n, 0, sd), m, n)
+     t(x + (y - apply(x, 1, mean)))
+ }
> X <- simdata(100, c(0, 1, 1, 1, 1, 1, 1, 1, 1, 1), 10)
> round(X[1:3, ], 3)
\end{Sinput}
\begin{Soutput}
       [,1]  [,2]   [,3]   [,4]   [,5]   [,6]   [,7]  [,8]    [,9]  [,10]
[1,] -6.558 0.925  0.312 10.095  2.894  3.519 -4.529 7.589  -6.712 -2.012
[2,] -0.638 0.573 -0.915  8.625  2.958  0.040 -2.599 5.552  -7.598 24.689
[3,] -5.917 7.251  3.721  8.961 -6.419 -4.051 -5.913 2.753 -10.037  4.057
\end{Soutput}
\begin{Sinput}
> y <- apply(X, 2, mean)
> round(y, 3)
\end{Sinput}
\begin{Soutput}
 [1] 0 1 1 1 1 1 1 1 1 1
\end{Soutput}
\end{Schunk}
The matrix $X=(x_{ij})$ above is of size $n\times m$ with $n=100$,
$m=10$. We consider that $X$ is a dataset of sample size $n$, and rows
$x_i = (x_{11},\ldots,x_{im})$, $i=1,\ldots,n$, are observations of a
random vector of $m$ dimensions. 

\subsection{Null Hypothesis}

Let $\mu$ be the unknown population mean of the row vectors. An
estimate of $\mu$ is the sample average of the rows defined as $y =
\bar x = \tfrac{1}{n}\sum_{i=1}^n x_i $. Let $f(\mu)$ be a 0/1-valued
(or false/true valued) function of $\mu$. The null hypothesis we are
going to test is represented as $f(\mu)=1$. For example, $f(\mu)=1$ if
$\mu_1$ is the largest among $\mu_1,\ldots,\mu_m$, and $f(\mu)=0$
otherwise. This $f(\mu)$ is implemented as {\tt mc1(mu)} below.
\begin{Schunk}
\begin{Sinput}
> mc1 <- function(x) all(x[1] >= x[-1])
> mc1(y)
\end{Sinput}
\begin{Soutput}
[1] FALSE
\end{Soutput}
\end{Schunk}
Although $f(y)=0$ gives a rough idea whether $f(\mu)=1$, we want to
calculate a real number ranging between 0 and 1 which indicates the
possibility of $f(\mu)=1$.
This is what {\tt scaleboot} calculates as $p$-values. 

\subsection{Bootstrap Probabilities}

A naive way to calculate $p$-value is the bootstrap resampling.  Let
$X^*=(x^*_{ij})$ be a bootstrap sample of $X$; each row $x^*_i$ is
obtained by resampling with replacement from the rows
$x_1,\ldots,x_n$.  Let $n'$ be the size of the resampling so that
$X^*$ is a matrix of size $n'\times m$. The bootstrap replicate is
$y^* = \bar x^* = \tfrac{1}{n'} \sum_{i=1}^{n'} x_i^*$. The following
code generates a $X^*$ of $n'=n$, and calculates $f(y^*)$. The
resampling is made via weight vector {\tt w}; $w_i$ is the number of
times that $x_i$ is resampled in $X^*$.
\begin{Schunk}
\begin{Sinput}
> countw <- function(x, w, fn) {
+     y <- apply(w * x, 2, sum)/sum(w)
+     fn(y)
+ }
> w <- as.vector(rmultinom(1, 100, rep(1, 100)))
> w
\end{Sinput}
\begin{Soutput}
  [1] 2 2 2 0 1 1 2 0 2 1 2 0 1 0 2 1 0 1 0 0 0 2 0 1 1 0 1 0 0 0 1 1 1 2 0 1 1
 [38] 2 1 0 1 1 1 0 2 1 1 1 0 0 1 1 0 2 1 0 2 1 1 1 3 1 0 2 1 3 0 0 0 0 1 1 2 1
 [75] 1 0 1 1 3 1 2 1 2 0 2 2 1 0 1 4 1 1 0 1 1 1 2 1 1 1
\end{Soutput}
\begin{Sinput}
> countw(X, w, mc1)
\end{Sinput}
\begin{Soutput}
[1] FALSE
\end{Soutput}
\end{Schunk}

Let $B$ be the number of bootstrap samples we will generate, and
$y^*_1,\ldots, y^*_B$ be the bootstrap replicates. Typically,
$B=10,000$. The BP is computed as $\sum_{i=1}^B f(y^*_i)/B$, where the
ordinary BP uses $n'=n$.  Since the first introduction by Felsenstein
(1985), it has been widely used as a $p$-value, but the bias is in
fact rather large.


\subsection{$P$-value Calculation}

{\tt scaleboot} calculates corrected $p$-values for improving BPs.
First load the package by
\begin{Schunk}
\begin{Sinput}
> library(scaleboot)
\end{Sinput}
\end{Schunk}

In the below, {\tt sa} specifies the scales, and {\tt nb} specifies
$B$ for each scale, so that $10,000\times13=130,000$ bootstrap samples
are generated internally.  It takes a few minutes on a pc.
\begin{Schunk}
\begin{Sinput}
> sa <- 9^seq(-1, 1, length = 13)
> nb <- 10000
> X.sb <- scaleboot(X, nb, sa, countw, mc1)
\end{Sinput}
\end{Schunk}
The result is shown by
\begin{Schunk}
\begin{Sinput}
> summary(X.sb)
\end{Sinput}
\begin{Soutput}
Raw Bootstrap Probability:  0.98 (0.10) 

Corrected P-values (percent):
       k.1         k.2          k.3          aic     
poly.1 0.07 (0.00)  0.07 (0.00)  0.07 (0.00) 3666.22 
poly.2 1.05 (0.05)  6.59 (0.35)  6.59 (0.35)  198.16 
poly.3 1.21 (0.05) 15.03 (0.95) 18.37 (1.30)   17.23 
sing.3 1.03 (0.04) 17.67 (0.74) 40.71 (1.47)  -18.05 

Best Model:  sing.3 
\end{Soutput}
\end{Schunk}

A class of AU $p$-values $p_k$ indexed by $k=1,2,3$, are calculated,
and they are labelled as {\tt k.1}, {\tt k.2}, and {\tt k.3}.  The
$p$-values are shown in percent, and the standard errors are given in
parentheses.  We should look at the row of {\tt sing.3} as indicated
as the best model in terms of AIC, and we can ignore the other
rows. $p_1\approx 1\%$ corresponds to the ordinary BP, and $p_2\approx
18\%$ corresponds to the AU $p$-value of Shimodaira (2002).  What we
recommend to use here is $p_3\approx41\%$; this is the AU $p$-value of
Shimodaira (2006).  For this particular example, the common practice
for calculating $p$-value is to use the multiple comparisons
method. If it applied to $y$, the $p$-value is $p=66\%$, which is
rather close to $p_3$ in our example, whereas $p_1$ is obviously too
small.

\subsection{Internal Steps}

We consider the following three steps (i)-(iii).
Internally, {\tt scaleboot} function (i) performs the multiscale
bootstrap, and (ii) estimates coefficients for candidate models. Then
the {\tt summary} method (iii) calculates the corrected $p$-values.
These steps are explained below.

The results of steps (i) and (ii) are shown by
\begin{Schunk}
\begin{Sinput}
> X.sb
\end{Sinput}
\begin{Soutput}
Multiscale Bootstrap Probabilities (percent):
1    2    3    4    5    6    7    8    9    10   11   12   13   
0.00 0.01 0.05 0.14 0.33 0.66 0.98 1.55 2.16 2.61 3.39 3.92 4.61 

Numbers of Bootstrap Replicates:
1     2     3     4     5     6     7     8     9     10    11    12    13    
10000 10000 10000 10000 10000 10000 10000 10000 10000 10000 10000 10000 10000 

Scales (Sigma Squared):
1      2      3      4      5      6      7 8     9     10   11    12   13    
0.1111 0.1603 0.2309 0.3333 0.4808 0.6944 1 1.449 2.083 3.03 4.348 6.25 9.091 

Coefficients:
       beta0           beta1           beta2            
poly.1 3.2017 (0.0182)                                  
poly.2 1.9077 (0.0217) 0.4004 (0.0069)                  
poly.3 1.6103 (0.0279) 0.6756 (0.0209) -0.0335 (0.0024) 
sing.3 0.9280 (0.0287) 1.3860 (0.0203)  1.0000 (0.0000) 

Model Fitting:
       rss     df pfit   aic     
poly.1 3690.22 12 0.0000 3666.22 
poly.2  220.16 11 0.0000  198.16 
poly.3   37.23 10 0.0001   17.23 
sing.3    1.95 10 0.9967  -18.05 

Best Model:  sing.3 
\end{Soutput}
\end{Schunk}

The results of (i) are the BPs for the 13 scales shown at first. Let
$\alpha_{\sigma^2}$ denote the BP at scale $\sigma^2$. Each BP is
calculated from 10,000 bootstrap samples of size $n'$ as the frequency
of observing $f(y^*)=1$. In {\tt scaleboot}, $n'$ is {\tt
round(n/sa[i])}, for $i=1,\ldots,13$. Then, the scale is recalculated
as $\sigma^2=n/n'$ for taking account of the discreteness. The actual
labor of {\tt scaleboot} function is in fact only step (i).

The step (ii) is performed by {\tt sbfit} function internally for
fitting parametric models to observed $\alpha_{\sigma^2}$'s.  By
default, four models are considered as candidates; {\tt poly.1}, {\tt
poly.2}, {\tt poly.3}, and {\tt sing.3}. Each of these models are
denoted as $\psi(\sigma^2|\beta)$. Let $z_{\sigma^2} = \Phi^{-1}( 1 -
\alpha_{\sigma^2})$ be the bootstrap $z$-value at scale $\sigma^2$,
where $\Phi^{-1}(p)$={\tt qnorm(p)}.  The coefficient vector $\beta$
is estimated by fitting $\sigma z_{\sigma^2} = \psi(\sigma^2 |
\beta)$.  Let $\hat\beta$ denote the estimated value; the detail of
fitting is explained later.  We may choose the model which minimizes
AIC value. The fitted curves are shown (Fig.~\ref{fig:diag1f}) by plotting
$\psi(\sigma^2|\hat\beta)$ as
\begin{Schunk}
\begin{Sinput}
> plot(X.sb, legend = "topleft")
\end{Sinput}
\end{Schunk}
\begin{figure}
\begin{center}
\includegraphics{usesb-011}
\caption{Model Fitting}\label{fig:diag1f}
\end{center}
\end{figure}
The same plot but in other variables can be shown
(Fig.~\ref{fig:diag1f2}) by, for example,
\begin{Schunk}
\begin{Sinput}
> plot(X.sb, xval = "sigma", log = "x", yval = "pvalue", legend = "topleft")
\end{Sinput}
\end{Schunk}
\begin{figure}
\begin{center}
\includegraphics{usesb-013}
\caption{Model Fitting ($x=\log \sigma$, $y=\alpha_{\sigma^2}$)}
\label{fig:diag1f2}
\end{center}
\end{figure}



{\tt poly.k} model is specified as a polynomial of $\sigma^2$;
$\psi(\sigma^2|\beta) = \sum_{j=0}^{k-1} \beta_j \sigma^{2j}$ for
$k\ge1$.  {\tt sing.k} model is specified as $\psi(\sigma^2|\beta) =
\beta_0 + \sum_{j=1}^{k-2} \beta_j \sigma^{2j} (1+a)/(1+a \sigma)$ for
$k\ge3$, where $a=\beta_{k-1}/(1-\beta_{k-1})$ so that $0\le a \le
\infty$ for $0\le\beta_{k-1}\le1$. The number {\tt k} for each model
denotes the number of coefficients in $\beta$.

The detail of model fitting is as follows. Let $B_i$ and $C_i$ be the
number of replicates and the observed number of times that $f(y^*)=1$,
respectively, for the bootstrap resampling of scale $\sigma^2_i$,
$i=1,\ldots,S$. Since each $C_i$ is distributed as binomial, the
log-likelihood is
\[
\ell(\beta) = \sum_{i=1}^S \Bigl\{
C_i \log \Phi(-\psi(\sigma^2_i|\beta)/\sigma_i)+
(B_i - C_i) \log \Phi(\psi(\sigma^2_i|\beta)/\sigma_i)
\Bigr\},
\]
where $\Phi(q)$={\tt pnorm(q)}. The estimate $\hat\beta$ is obtained
by maximizing $\ell(\beta)$ numerically.  The goodness of fit is
measured by the difference of AIC values between the specified model
and an unconstrained binomial model;
\[
{\rm AIC} = ( -2\ell(\hat\beta)+2k ) - (-2\hat\ell + 2S),
\]
where $\hat\ell = \sum_{i=1}^S ( C_i\log(C_i/B_i) +
(B_i-C_i)\log(1-C_i/B_i)) $.


The step (iii) is performed by the the {\tt summary} method as already
mentioned. The first line shows the ``raw'' BP $\alpha_1$ (the BP
obtained from the ordinary bootstrap resampling).  The main results
are the corrected $p$-values follow next. For each model, we calculate
$q_k$, $k=1,2,3$, by
\[
q_k = \sum_{j=0}^{k-1} \frac{(-1-\sigma_0^2)^j}{j!}
\frac{\partial^j \psi(\sigma^2|\hat\beta)}{\partial (\sigma^2)^j}
\Bigr|_{\sigma_0^2}.
\]
Then the corrected $p$-values are calculated by $p_k=1-\Phi(q_k)$.
By default $\sigma_0^2=1$. The
calculation of $q_k$ is interpreted as extrapolation of $\sigma
z_{\sigma^2}$ to $\sigma^2=-1$ by using the first $k$ terms of the
Taylor series. According to the theory of Shimodaira (2006), the
unbiased $p$-value is, if exists, obtained by taking the limit
$k\to\infty$. The extrapolated curves are shown (Fig.~\ref{fig:diag1s}) by
\begin{Schunk}
\begin{Sinput}
> plot(summary(X.sb), legend = "topleft")
\end{Sinput}
\end{Schunk}
\begin{figure}
\begin{center}
\includegraphics{usesb-015}
\caption{Extrapolation}\label{fig:diag1s}
\end{center}
\end{figure}


\section{Hierarchical Clustering}

\subsection{Pvclust Package}

{\tt scaleboot} package includes an interface to {\tt pvclust} package
(Suzuki and Shimodaira 2006) of R for bootstrapping hierarchical
clustering.  We use {\tt pvclust} to calculate multiscale BPs for
clusters, from which we calculate an improved version of AU p-values
using {\tt scaleboot}. See {\tt help(lung73)} for further details of
the following example.

\subsection{Using Pvclust}

We work on {\tt lung} dataset (Garber et al.~2001) included in {\tt
pvclust}.  It is a DNA microarray dataset of 73 lung tissues (arrays)
with 916 observations of genes.  To draw dendrograms in terms of the
arrays, we resample genes in our analysis; this may be interpreted as
assessing the uncertainty due to variability of genes.  The function
{\tt pvclust} first obtains a dendrogram by a hierarchical clustering
method, and then calculates the multiscale BPs for each cluster of the
dendrogram.
\begin{Schunk}
\begin{Sinput}
> library(pvclust)
> data(lung)
> sa <- 9^seq(-1, 1, length = 13)
> nb <- 10000
> lung73.pvclust <- pvclust(lung, r = 1/sa, nboot = nb)
\end{Sinput}
\end{Schunk}
The above code may take a day, so it would be a good idea to run with
nb=1000 so that 10 times faster. However, nb=1000 should be just
for checking the program, and nb=10,000 (at least) is recommended for
publishing the result. 

\subsection{Model Fitting}

We next apply {\tt sbfit} function of {\tt scaleboot} to the
multiscale BPs. For each cluster of the dendrogram, parametric models
are fitted to the BPs.
\begin{Schunk}
\begin{Sinput}
> library(scaleboot)
> lung73.sb <- sbfit(lung73.pvclust)
\end{Sinput}
\end{Schunk}

\subsection{Lung73 Dataset}

The results of the previous two sections ({\tt lung73.pvclust} and
{\tt lung73.sb}) are in fact stored in {\tt lung73} dataset of {\tt
scaleboot}. For users who want to try the examples, just type as follows.
\begin{Schunk}
\begin{Sinput}
> library(scaleboot)
> data(lung73)
\end{Sinput}
\end{Schunk}

We have used a cluster computer of 40 cpus for parallel
computing using {\tt snow} package.
The following code may take only an hour.
\begin{Schunk}
\begin{Sinput}
> library(snow)
> cl <- makeCluster(40)
> library(pvclust)
> data(lung)
> sa <- 9^seq(-1, 1, length = 13)
> nb <- 10000
> lung73.pvclust <- parPvclust(cl, lung, r = 1/sa, nboot = nb)
> library(scaleboot)
> lung73.sb <- sbfit(lung73.pvclust, cluster = cl)
\end{Sinput}
\end{Schunk}

\subsection{$P$-value Calculation}

To calculate AU p-values ($p_3$) from {\tt lung73.sb} and write them
back to {\tt lung73.pvclust}, we do
\begin{Schunk}
\begin{Sinput}
> lung73.k3 <- sbpvclust(lung73.pvclust, lung73.sb)
\end{Sinput}
\end{Schunk}
To see the results, we simply plot the dendrogram
(Fig.~\ref{fig:lung73tree}) by
\begin{Schunk}
\begin{Sinput}
> library(pvclust)
> plot(lung73.k3, cex = 0.5, cex.pv = 0.7)
> pvrect(lung73.k3)
\end{Sinput}
\end{Schunk}
\begin{figure}
\begin{center}
\includegraphics{usesb-022}
\caption{Dendrogram of lung73 dataset}\label{fig:lung73tree}
\end{center}
\end{figure}
To calculate $p_2$ instead of $p_3$, specify {\tt k=2},
\begin{Schunk}
\begin{Sinput}
> lung73.k2 <- sbpvclust(lung73.pvclust, lung73.sb, k = 2)
\end{Sinput}
\end{Schunk}

\subsection{Diagnostics of Fitting}

The fitted curves are drawn by the plot method. For node 67, say, a
plot with legend is obtained (Fig.~\ref{fig:lungplot67}) by
\begin{Schunk}
\begin{Sinput}
> plot(lung73.sb[[67]], legend = "topleft")
\end{Sinput}
\end{Schunk}
\begin{figure}
\begin{center}
\includegraphics{usesb-025}
\caption{Model fitting for node 67}\label{fig:lungplot67}
\end{center}
\end{figure}
All the calculated $p$-values for node 67 are given by
\begin{Schunk}
\begin{Sinput}
> summary(lung73.sb[[67]])
\end{Sinput}
\begin{Soutput}
Raw Bootstrap Probability:  3.63 (0.19) 

Corrected P-values (percent):
       k.1          k.2          k.3          aic      
poly.1 18.41 (0.10) 18.41 (0.10) 18.41 (0.10) 52878.29 
poly.2  5.46 (0.09) 83.51 (0.32) 83.51 (0.32)  1356.63 
poly.3  3.95 (0.08) 86.05 (0.29) 92.56 (0.28)   464.71 
sing.3  3.31 (0.07) 77.02 (0.47) 95.10 (0.17)    25.11 

Best Model:  sing.3 
\end{Soutput}
\end{Schunk}
The extrapolation using the best model is shown
(Fig.~\ref{fig:lungext67}) by
\begin{Schunk}
\begin{Sinput}
> plot(summary(lung73.sb[[67]]), legend = "topleft")
\end{Sinput}
\end{Schunk}
\begin{figure}
\begin{center}
\includegraphics{usesb-028}
\caption{Extrapolation for node 67}\label{fig:lungext67}
\end{center}
\end{figure}

For a set of nodes, $p$-values are given by
\begin{Schunk}
\begin{Sinput}
> summary(lung73.sb[c(62, 67, 69, 71)])
\end{Sinput}
\begin{Soutput}
Corrected P-values (percent):
   raw          k.1          k.2          k.3          model  aic    
62 95.68 (0.20) 95.92 (0.10) 98.64 (0.10) 98.61 (0.12) poly.3 -12.01 
67  3.63 (0.19)  3.31 (0.07) 77.02 (0.47) 95.10 (0.17) sing.3  25.11 
69 29.49 (0.46) 29.65 (0.17) 75.37 (0.22) 75.83 (0.34) poly.3 -14.09 
71 25.20 (0.43) 25.95 (0.17) 84.44 (0.18) 85.91 (0.27) poly.3  11.49 
\end{Soutput}
\end{Schunk}
Also plots are shown (Fig.~\ref{fig:lung73nodes}) by
\begin{Schunk}
\begin{Sinput}
> plot(lung73.sb[c(62, 67, 69, 71)])
\end{Sinput}
\end{Schunk}
\begin{figure}
\begin{center}
\includegraphics{usesb-031}
\caption{Model fitting for a set of nodes} \label{fig:lung73nodes}
\end{center}
\end{figure}

\section{Phylogenetic Inference}

\subsection{CONSEL Software}

{\tt scaleboot} has a front end for phylogenetic inference, and it may
eventually replace {\tt CONSEL} software (Shimodaira and Hasegawa
2001) for testing phylogenetic trees. Currently, {\tt scaleboot} does
not have a method for file conversion of several phylogenetic
software, and so we must use {\tt CONSEL} for this purpose before
applying {\tt scaleboot} to calculate an improved version of AU
p-values for trees and edges.  See {\tt help(mam15)} for further
details of the following example.

\subsection{Mammal Dataset}

We work on an example of phylogenetic analysis of six mammal species:
Homo sapiens (human), Phoca vitulina (harbor seal), Bos taurus (cow),
Oryctolagus cuniculus (rabbit), Mus musculus (mouse), Didelphis
virginiana (opossum). The dataset was originally used in Shimodaira
and Hasegawa (1999). 

For Unix users, download {\tt mam15-files.tgz}, and for Windows users
download {\tt mam15-files.zip}.  The details of dataset files are as
follows.  {\tt mam15.aa}: amino acid sequences ($n=3414$) of mtDNA for
the six mammals.  {\tt mam15.ass}: association vectors for edges and
trees.  {\tt mam15.lnf}: site-wise log-likelihood values (output from
PAML).  {\tt mam15.log}: detailed information for the associations.
{\tt mam15.mt}: site-wise log-likelihood values (output from seqmt).
{\tt mam15.tpl}: 15 tree topologies.

\subsection{Likelihood Calculation of Trees}

The main body of the dataset is the amino acid sequences ({\tt mam15.aa}).
We  consider $m=15$ tree topologies of the six mammals ({\tt mam15.tpl});
\begin{verbatim}  
((Homsa,(Phovi,Bosta)),Orycu,(Musmu,Didvi)); t1
(Homsa,Orycu,((Phovi,Bosta),(Musmu,Didvi))); t2
(Homsa,((Phovi,Bosta),Orycu),(Musmu,Didvi)); t3
(Homsa,(Orycu,Musmu),((Phovi,Bosta),Didvi)); t4
((Homsa,(Phovi,Bosta)),(Orycu,Musmu),Didvi); t5
(Homsa,((Phovi,Bosta),(Orycu,Musmu)),Didvi); t6
(Homsa,(((Phovi,Bosta),Orycu),Musmu),Didvi); t7
(((Homsa,(Phovi,Bosta)),Musmu),Orycu,Didvi); t8
(((Homsa,Musmu),(Phovi,Bosta)),Orycu,Didvi); t9
(Homsa,Orycu,(((Phovi,Bosta),Musmu),Didvi)); t10
(Homsa,(((Phovi,Bosta),Musmu),Orycu),Didvi); t11
((Homsa,((Phovi,Bosta),Musmu)),Orycu,Didvi); t12
(Homsa,Orycu,(((Phovi,Bosta),Didvi),Musmu)); t13
((Homsa,Musmu),Orycu,((Phovi,Bosta),Didvi)); t14
((Homsa,Musmu),((Phovi,Bosta),Orycu),Didvi); t15
\end{verbatim}
The maximum likelihood estimates for these trees are calculated by
PAML (Yang 1997).  Let $x_{ij}$ be the site-wise log-likelihood for
sites $i=1,\ldots,n$, and trees $j=1,\ldots,m$. The log-likelihood of
tree-$j$ is $\sum_{i=1}^n x_{ij}$. The large $n$ justifies the central
limit theorem for $y=\bar x$, and allows us to resample $x_{ij}$
directly without recalculation of the maximum likelihood
estimates. The matrix $X=(x_{ij})$ is produced by PAML and stored in
{\tt mam15.lnf}.  It is converted by CONSEL to a simpler format and
stored in {\tt mam15.mt}. The command is

\verb!seqmt --paml mam15.lnf!


\subsection{$P$-value Calculation for Trees}

The AU $p$-values for trees are calculated simply by
\begin{Schunk}
\begin{Sinput}
> library(scaleboot)
> mam15.mt <- read.mt("mam15.mt")
> mam15.trees <- relltest(mam15.mt)
> summary(mam15.trees)
\end{Sinput}
\end{Schunk}
The {\tt relltest} function above may take a half hour. The next
section can be skipped if only tree selection is of interest.

\subsection{$P$-value Calculation for Clusters}

We can also calculate AU $p$-values for clusters (edges) of trees.  We
have to know, for each cluster, in which of the 15 trees it is
included.  The file {\tt mam15.ass} has this information, which was
generated using CONSEL by the command 

\verb!treeass --outgroup 6 mam15.tpl > mam15.log!

It also produces {\tt mam15.log} for human readable information.  A
part of {\tt mam15.log} is as follows.
\begin{verbatim}
# leaves: 6
6
  1 Homsa
  2 Phovi
  3 Bosta
  4 Orycu
  5 Musmu
  6 Didvi

# base edges: 10
10 6
          
    123456
  1 +++---  ; (0.2000)
  2 ++++--  ; (0.2000)
  3 +--+--  ; (0.2000)
  4 -+++--  ; (0.2000)
  5 ---++-  ; (0.2000)
  6 +--++-  ; (0.2000)
  7 -++++-  ; (0.2000)
  8 +++-+-  ; (0.2000)
  9 +---+-  ; (0.2000)
 10 -++-+-  ; (0.2000)
\end{verbatim}
The above defines clusters (edges) named e1,...e10. For example, e1 =
{\tt +++---} = (Homsa, Phovi, Bosta).

The AU $p$-values for clusters as well as trees are calculated simply by
\begin{Schunk}
\begin{Sinput}
> library(scaleboot)
> mam15.mt <- read.mt("mam15.mt")
> mam15.ass <- read.ass("mam15.ass")
> mam15.relltest <- relltest(mam15.mt, ass = mam15.ass)
> summary(mam15.relltest)
\end{Sinput}
\end{Schunk}

\subsection{Mam15 Dataset}

The results of the previous sections (
{\tt mam15.mt}, {\tt mam15.ass}, and
{\tt mam15.relltest}) are in fact stored in {\tt mam15} dataset of {\tt
scaleboot}. For users who want to try the examples, just type as follows.
\begin{Schunk}
\begin{Sinput}
> library(scaleboot)
> data(mam15)
\end{Sinput}
\end{Schunk}

The results for trees are extracted by
\begin{Schunk}
\begin{Sinput}
> mam15.trees <- mam15.relltest[1:15]
\end{Sinput}
\end{Schunk}

We have used a cluster computer of 40 cpus for parallel computing
using {\tt snow} package.  The following code may take only 10
minutes, although we have used the number of resamples ten times
larger than the default value.
\begin{Schunk}
\begin{Sinput}
> library(snow)
> cl <- makeCluster(40)
> library(scaleboot)
> mam15.mt <- read.mt("mam15.mt")
> mam15.ass <- read.ass("mam15.ass")
> mam15.relltest <- relltest(mam15.mt, nb = 1e+05, ass = mam15.ass)
\end{Sinput}
\end{Schunk}

\subsection{Interpreting the Results}

First we sort the results in increasing order of log-likelihood difference,
\begin{Schunk}
\begin{Sinput}
> stat <- attr(mam15.trees, "stat")
> o <- order(stat)
> mam15.trees <- mam15.trees[o]
> summary(mam15.trees)
\end{Sinput}
\begin{Soutput}
Corrected P-values (percent):
    raw          k.1          k.2          k.3          model  aic     
t1  57.58 (0.16) 56.16 (0.04) 74.55 (0.05) 74.55 (0.05) poly.2  964.33 
t3  31.86 (0.15) 30.26 (0.05) 46.41 (0.09) 45.33 (0.13) poly.3 1306.50 
t2   3.68 (0.06)  3.68 (0.03) 12.97 (0.20) 16.12 (0.45) sing.3   -6.21 
t5   1.34 (0.04)  1.33 (0.02)  7.92 (0.25) 10.56 (0.56) sing.3  -14.11 
t6   3.18 (0.06)  3.15 (0.02) 13.15 (0.21) 15.86 (0.44) sing.3   -2.49 
t7   0.49 (0.02)  0.52 (0.01)  3.66 (0.21)  4.75 (0.42) sing.3  -12.04 
t4   1.55 (0.04)  1.53 (0.02) 10.54 (0.27) 14.84 (0.66) sing.3   -7.57 
t15  0.08 (0.01)  0.07 (0.00)  1.11 (0.19)  1.85 (0.48) sing.3  -17.08 
t8   0.00 (0.00)  0.00 (0.00)  0.04 (0.03)  0.07 (0.07) sing.3  -13.68 
t14  0.22 (0.01)  0.23 (0.01)  2.76 (0.26)  4.59 (0.71) sing.3  -10.79 
t13  0.02 (0.00)  0.01 (0.00)  0.50 (0.20)  1.30 (0.83) sing.3  -15.14 
t9   0.00 (0.00)  0.00 (0.00)  0.23 (0.05)  1.41 (0.29) sing.3  -15.86 
t11  0.00 (0.00)  0.00 (0.00)  0.00 (0.00)  0.00 (0.00) poly.3  -19.71 
t10  0.00 (0.00)  0.00 (0.00)  0.00 (0.00)  0.00 (0.00) poly.3  -17.27 
t12  0.00 (0.00)  0.00 (0.00)  0.00 (0.00)  0.00 (0.00) poly.3  -19.61 
\end{Soutput}
\end{Schunk}

Next we look at $p$-values. We confirm that $p_1$ (the second column)
is almost the same as the raw BP (the first column); this should be so
if the model fitting is good. Only two trees, i.e., t1 and t3, have
$p_1>0.05$. It is known that the bias of $p_1$ is large so that often
leads to false positives for tree selection. $p_2$ improves $p_1$ by
correcting the bias. Six trees, i.e., t1, t3, t2, t5, t6, and t4, have
$p_2>0.05$. $p_3$ improves $p_2$ even more, although the trees of
$p_3>0.05$ are the same six trees in this example.


Finally we examine model fitting.  According to AIC values, the
fitting is good overall except for the top two trees; however note
that the AIC values should be about 10 times smaller if the default
nb=10,000 were used.  The fitting curves for the top four trees are
shown (Fig.~\ref{fig:mamplots}) by
\begin{Schunk}
\begin{Sinput}
> plot(mam15.trees[1:4])
\end{Sinput}
\end{Schunk}
\begin{figure}
\begin{center}
\includegraphics{usesb-040}
\caption{Model fitting for the top four trees}\label{fig:mamplots}
\end{center}
\end{figure}
According to the plots, the fitting is rather good even to the top two
trees. 


\begin{thebibliography}{9}

\bibitem{bib:Fels:85:CLP}
Felsenstein, J.
(1985).
\newblock Confidence limits on phylogenies: an approach using the bootstrap.
\newblock {\em Evolution} {\bf 39} 783--791.

\bibitem{bib:Garber2001}
  Garber, M. E. et al. (2001)
\newblock Diversity of gene expression in adenocarcinoma of the lung.
\newblock \emph{Proceedings of the National Academy of Sciences}
  {\bf 98} 13784--13789 (dataset is available from
  \url{http://genome-www.stanford.edu/lung_cancer/adeno/}).

\bibitem{bib:Shim:2002:AUT}
Shimodaira, H.
(2002).
\newblock An approximately unbiased test of phylogenetic tree selection.
\newblock {\em Syst. Biol.} {\bf 51} 492--508.

\bibitem{bib:Shimo:2006:AUT}
Shimodaira, H.
(2006)
\newblock Approximately Unbiased Tests for Singular
Surfaces via Multiscale Bootstrap Resampling.
\newblock Research Report {B}-430, Dept. Mathematical and Computing
  Sciences, Tokyo Institute of Technology, Tokyo.

\bibitem{bib:SH1999} 
Shimodaira, H. and Hasegawa, M. (2001)
\newblock CONSEL: for assessing the
  confidence of phylogenetic tree selection.
\newblock \emph{Bioinformatics} {\bf 17} 1246--1247 (software is available from
  \url{http://www.is.titech.ac.jp/~shimo/prog/consel/}).

\bibitem{bib:SS2006}
Suzuki, R. and Shimodaira, H. (2006)
\newblock pvclust: An R package for hierarchical clustering with p-values.
\newblock \emph{Bioinformatics} {\bf 22} 1540--1542 (software is available from
\url{http://www.is.titech.ac.jp/~shimo/prog/pvclust/}).

\bibitem{bib:Yang1997}
Yang, Z. (1997)
\newblock  PAML: a program package for phylogenetic analysis by
  maximum likelihood.
\newblock \emph{Computer Applications in BioSciences}
{\bf 13} 555--556 (software is available from
  \url{http://abacus.gene.ucl.ac.uk/software/paml.html}).


\end{thebibliography}

\end{document}
